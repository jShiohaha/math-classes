\documentclass[12pt]{article}
\usepackage[margin=1in]{geometry} 
\usepackage{amsmath,amsthm,amssymb,amsfonts,stmaryrd}
\usepackage{enumitem}
\usepackage{tabu}
\usepackage{fixltx2e}
\usepackage{xcolor}
 
\newcommand{\N}{\mathbb{N}}
\newcommand{\Z}{\mathbb{Z}}
 
\newenvironment{problem}[2][Problem]{\begin{trivlist}
\item[\hskip \labelsep {\bfseries #1}\hskip \labelsep {\bfseries #2.}]}{\end{trivlist}}
%If you want to title your bold things something different just make another thing exactly like this but replace "problem" with the name of the thing you want, like theorem or lemma or whatever

 
\begin{document}
%% TODO: Check for statement of existence of variables and restraints on variables
 
\title{Math 310 Homework 5}
\author{Jacob Shiohira}
\maketitle

\noindent
\textit{Note:} This homework took a total of 6 hours. I initially did it alone, but I did review with Jacob Warner.

\begin{problem}{1} Non-book problem \\

% TODO: DO WE HAVE TO DEFINE A RELATION ON S?
\noindent
\textbf{Proposition} Let $S=\{(a,b) | a,b \in \Z$ and $b \neq 0\}$. Let $(a,b) \sim (c,d)$ if and only if $ad=bc$. Prove that $ \sim $ is an equivalence relation on $S$. \\

\noindent
Let $S=\{(a,b) | a,b \in \Z$ and $b \neq 0\}$. In order to prove that $ \sim $ is an equivalence relation, we must show that it retains the reflexive, symmetric, and transitive properties.
\begin{enumerate}
\item Reflexive: For all $a,b \in \Z (b \neq 0)$, let $(a,b)$ be an ordered pair. Multiplication of integers is commutative, so $(a,b), (a,b) \Longleftrightarrow ab=ba$. This shows $S$ is reflexive.
\item Symmetric: For all $a,b,c,d \in \Z (bd \neq 0)$, let $(a,b)$ and $(c,d)$ be ordered pairs. So, by the commutative property of multiplication, $(a,b) \sim (c,d) \Longleftrightarrow ad=bc \Longleftrightarrow cb=da \Longleftrightarrow (a,b)  (c,d) \sim (a,b)$. This shows $S$ is symmetric. 
\item Transitive: For all $a,b,c,d, e, f \in \Z (bdf \neq 0)$, let $(a,b)$, $(c,d)$, and $(e,f)$ be ordered pairs. If $(a,b) \sim (c,d)$ and $(c,d) \sim (e,f)$, then $ad=bc$ and $cf=de$. Multiplying yields $adcf=bcde$. By the associativity of multiplication, $af(cd)=be(cd)$. Suppose $cd \neq 0$. Then, we can divide both sides by $cd$. Then, $af=be$ and $(a,b) \sim (e,f)$, so $S$ is transitive.
\end{enumerate}

\noindent
We see that since $\sim$ satisfies all three properties, it is in fact an equivalence relation. \qedsymbol

% \noindent
% Let $S=\{(a,b) | a,b \in \Z$ and $b \neq 0\}$. Since the proposition contains a biconditional statement, we will proceed by first showing that the forward implication holds, and % then show that the backward implication holds. The forward implication says 
% \begin{center}
% If $(a,b) \sim (c,d)$, then $ad=bc$.
% \end{center}
% \noindent
% Suppose $(a,b) \sim (c,d)$. We then want to show that $ad=bc$. According to our assumption, we know $(a,b) \sim (c,d)$, and we know that the three properties of equivalence relations hold: reflexive, symmetric, and transitive. The symmetric property says there are ordered pairs of positive integers $(a,b)$ and $(c,d)$ such that $(a,b) \sim (c,d)$, which is equivalent to $((a,b), (c,d)) \in S$. Then, $ad=bc$. This equation is equivalent to $cb=da$, so $(c,d) \sim (a,b)$. 


% The backward implication says 
% \begin{center}
% If $ad=bc$, then $(a,b) \sim (c,d)$. 
% \end{center}
% \noindent
%% TODO: is this worded correctly?
% Well, by the commutative property of multiplication, we know that for an ordered pair of positive integers, $(a,b)$, $ab=ba$. Thus, $((a,b), (a,b)) \in S$. So, we see that $(a,b) \sim (a,b)$ is reflexive. Now, let $(a,b) \in S$ and $(c,d) \in S$. Then, our assumption, $ad=bc$, is a consequence of symmetry such that ... $(a,b) \sim (c,d)$, which is equivalent to $((a,b), (c,d)) \in S$. Then, $ad=bc$. This equation is equivalent to $cb=da$, so $(c,d) \sim (a,b)$. this shows $S$ is symmetric. 
\end{problem}

\begin{problem}{2} Section $2.1$ $\#3$ \\

\noindent
Every published book has a ten-digit ISBN-10 number that is usually in the form $x_1$-$x_2x_3x_4$-$x_5x_6x_7x_8x_9$-$x_{10}$ (where each $x_i$ is a single digit). The first $9$ digits identify the book. The last digit $x_{10}$ is a check digit. It is chosen so that
\begin{center}
$10x_1+9x_2+8x_3+7x_4+6x_5+5x_6+4x_7+3x_8+2x_9+1x_{10} \equiv 0 (\text{mod}11)$.
\end{center}
\noindent
If an error is made when scanning or keying an ISBN number into a computer, the left side of the congruence will not be congruent to $0$ modulo $11$, and the number will be rejected as invalid. Which of the following are apparently valid ISBN numbers? (\textit{Note}: Treat the letter $X$ as if it were the number $10$.)
\begin{enumerate}[=(\alph*)]
\item $3$-$540$-$90518$-$9$
\begin{align*}
10(3)+9(5)+8(4)+7(0)+6(9)+5(0)+4(5)+3(1)+2(8)+1(9) & \equiv 0 (\text{mod}11) \\
30+45+32+0+54+0+20+3+16+9 & \equiv 0 (\text{mod}11) \\
209 & \equiv \\
\end{align*}
\noindent
$209$ is divisible by $11$, so $3$-$540$-$90518$-$9$ is a valid ISBN.

\item $0$-$031$-$10559$-$5$
\begin{align*}
10(0)+9(0)+8(3)+7(1)+6(1)+5(0)+4(5)+3(5)+2(9)+1(5) & \equiv 0 (\text{mod}11) \\
0+0+24+7+6+0+20+15+18+5 & \equiv \\
95 & \equiv \\
\end{align*}
\noindent
$95$ is not divisible by $11$, so $0$-$031$-$10559$-$5$ is not a valid ISBN.

\item $0$-$385$-$49596$-$X$
\begin{align*}
10(0)+9(3)+8(8)+7(5)+6(4)+5(9)+4(5)+3(9)+2(6)+1(X) & \equiv 0 (\text{mod}11) \\
0+27+64+35+24+45+20+27+12+10 & \equiv 0 (\text{mod}11) \\
264 & \equiv \\
\end{align*}
\noindent
$264$ is divisible by $11$, so $0$-$385$-$49596$-$X$ is a valid ISBN.

\end{enumerate}
\end{problem}

\begin{problem}{3} Section $2.1$ $\#6$ \\

\noindent
\textbf{Proposition}: If $a \equiv b(\text{mod }n)$ and $k|n$, is it true that $a \equiv b(\text{mod }k)$? Justify your answer. 
\vspace{.3cm}

\noindent
As per a normal implication, we will assume both $a \equiv b(\text{mod }n)$ and $k|n$ are true and try to show that $a \equiv b(\text{mod }k)$. From the definition of congruence, we know that $n|a-b$. Then, the definition of divisibility tells us that $a-b=np$ and $n=kq$ for integers $p,q$. By substituting for $n$,
\begin{center}
$a-b=(kq)p$.
\end{center}
\noindent
By associativity of multiplication,
\begin{center}
$a-b=k(qp)$.
\end{center}
\noindent
We now see that $k|a-b$ is also true. So, $a \equiv b (\text{mod }k)$. \qedsymbol
\end{problem}

\begin{problem}{4} Section $2.1$ $\#13$ \\

\noindent
\textbf{Proposition}: $a \equiv b(\text{mod }n)$ if and only if $a$ and $b$ leave the same remainder when divided by $n$.

\vspace{.3cm}

\noindent
Since the proposition contains a biconditional statement, we will proceed by first showing that the forward implication holds, and then show that the backward implication holds. The forward implication says 
\begin{center}
If $a \equiv b(\text{mod }n)$, then $a$ and $b$ leave the same remainder when divided by $n$.
\end{center}
\noindent
As for any implication, we assume that $a \equiv b(\text{mod }n)$ is true. We then want to show $a$ and $b$ leave the same remainder when divided by $n$ is also true. By definition of congruence, we know that $n|a-b$ and the definition of the divisibility yields $a-b=nk$ for some integer $k$. 
%% TODO: Can I say this? vv
Then, by the Fundamental Theorem of Arithmetic, we know if $n|a-b$, then $n|b$ and $n|a$. By the division algorithm, we know that when a number $x$ divides another $y$, there exist unique integers $q,r$ such that 
\begin{center}
$y=xq+r$, $0 \leq r < x$.
\end{center}
\noindent
Then for integers $q_1, q_2, r_1, r_2$,
\begin{center}
$a=nq_1+r_1$, $0 \leq r_1 < n$, \\
$b=nq_2+r_2$, $0 \leq r_2 < n$.
\end{center}
\noindent
We can then reference $a-b=nk$ that we stated earlier and substitute our new values of $a$ and $b$,
\begin{center}
$nq_1+r_1-nq_2-r_2=nk$.
\end{center}
\noindent
By rearranging the equation,
\begin{center}
$r_1-r_2=nk-nq_1+nq_2$.
\end{center}
\noindent
By factoring of addition,
 \begin{center}
$r_1-r_2=n(k-q_1+q_2)$.
\end{center}
\noindent
Since the integers are closed under addition and multiplication, $k-q_1+q_2$ is an integer. By the definition of divisibility, we know that $n$ divides $r_1-r_2$. Well, $r_1$ and $r_2$ are both strictly less than $n$, and $r_2-r_1$ will be strictly less than $n$. So, the only way that $n$ divides $r_1-r_2$ is if $r_1-r_2=0$. Thus, we have that $r_1=r_2$, and $a$ and $b$ leave the same remainder when divided by $n$. 

\vspace{.3cm}

\noindent
The backward implication says 
\begin{center}
If $a$ and $b$ leave the same remainder when divided by $n$, then $a \equiv b(\text{mod}n)$.
\end{center}
\noindent
Again, we assume that $a$ and $b$ leave the same remainder when divided by $n$ is true. We then want to show that $a \equiv b(\text{mod }n)$ is also true. By the division algorithm, we know that division leaves a remainder value,
\begin{center}
$a$ divided by $n$: $a=np+r$ for some integer $p$, \\
$b$ divided by $n$: $b=nq+r$ for some integer $q$.
\end{center}
\noindent
However, we know that both $a$ divided by $n$ and $b$ divided by $n$ yield the same remainder. So, we can rearrange both equations for $r$,
\begin{center}
$a-np=b-nq$.
\end{center}
By rearranging,
\begin{center}
$a-b=np-nq$.
\end{center}
\noindent
By factoring,
\begin{center}
$a-b=n(p-q)$.
\end{center}
\noindent
Since the integers are closed under addition and multiplication, $p-q$ is an integer. Thus, $n|a-b$ and we have that $a \equiv b(\text{mod }n)$.
\end{problem}

\begin{problem}{5} Section $2.1$ $\#14$ \\

\noindent
\begin{enumerate}[label=(\alph*)]
\item Prove or disprove: If $ab \equiv 0 (\text{mod}n)$, then $a \equiv 0 (\text{mod}n)$ or $b \equiv 0 (\text{mod}n)$. \\

\noindent
We will disprove by giving a single counter-example, which is sufficient because we are showing a single instance in which the proposition does not hold. Thus, it cannot hold for for all integers. By the definition of congruence, $ab-0=np$ for some integer $p$. Consider $a=5,b=6,n=10$,
\begin{align*}
ab-0 & =np, \\
(5)(6)-0 & =(10)p, \\
30-0 & =10p, \\
30 & =10p.
\end{align*}
\noindent
We see that $p=3$ with a remainder of $0$. Let us now consider $a \equiv 0 (\text{mod}n)$ and $b \equiv 0 (\text{mod}n)$. Again, by the definition of congruence, $a-0=np$ and $b-0=nq$ for some integers $p,q$. However,

\begin{align*}
a-0 & =np, \\
5-0 & =(10)p, \\
5 & =10p.
\end{align*}
\noindent
and
\begin{align*}
b-0 & =np, \\
6-0 & =(10)p, \\
6 & =10p.
\end{align*}
\noindent
We see that there is no integer value $p$ that can satisfy the final equations. Thus, we must reference the division algorithm and consider remainders of $r_1=5$ when $a=5$ and $r_2=6$ when $b=6$. Thus, we have disproved the proposition. \\

\item Do part $(a)$ when $n$ is prime. \\

If $ab \equiv 0 (\text{mod }n)$, we know $ab-0=nk$ for some integer $k$ by the definition of congruence. We can simplify $ab-0$ to just $ab$. By Theorem $1.5$, if $ab=nk$, which is equivalent to $n|ab$, then either $n|a$ or $n|b$. We then proceed in two cases, \\

If $n|a$, then, by the definition of divisibility, we know $a=np$ for some integer $p$. Then, let us subtract $0$ from both sides, $a-0=np-0$. The $0$ on the RHS can be considered taken away since it has no effect, and we are left with $a-0=np$. By the definition of congruence, we have $a \equiv 0 (\text{mod }n)$. \\

If $n|b$, then, by the definition of divisibility, we know $b=nq$ for some integer $q$. Then, let us subtract $0$ from both sides, $b-0=nq-0$. The $0$ on the RHS can be considered taken away since it has no effect, and we are left with $b-0=nq$. By the definition of congruence, we have $b \equiv 0 (\text{mod }n)$. \\

We have seen that if $ab \equiv 0 (\text{mod }n)$ and $n$ is prime, either $a \equiv 0 (\text{mod }n)$ or $b \equiv 0 (\text{mod }n)$. This proves our proposition. \qedsymbol
\end{enumerate}
\end{problem}

\begin{problem}{6} Section $2.1$ $\#21$ \\

\noindent
\begin{enumerate}[label=(\alph*)]
\item Show that $10^n \equiv 1 (\text{mod}9)$ for every positive $n$. \\

\noindent
Choose an arbitrary integer $n$. Consider the equality, 
\begin{center}
$10^n \equiv 1 (\text{mod}9)$.
\end{center}
We know that $10 \equiv 1 (\text{mod }9)$ by the definition of congruence $9|10-1$. Then, we can cite Theorem $2.2$ because our congruence turns into $10^n \equiv 1^n (\text{mod }9)$. \qedsymbol

\item Prove that every positive integer is congruent to the sum of its digital mod $9[$ for example, $38 \equiv 11 (\text{mod}9)]$. \\

\noindent
We proved in part $a$ that $10^n \equiv 1 (\text{mod }9)$. So, if we could figure out a way to represent integers with terms of $10^n$, where $n$ is an integer, we could prove that the sum of its digits is congruent to the sum mod $9$. So, in the example of $38$, it can be expanded as a number represented by coefficients of powers of 10 by $8 \cdot 10^0 + 3 \cdot 10^1$. Then, $11=1 \cdot 10^0 + 1 \cdot 10^1$, and the congruence would be $8 \cdot 10^0 + 3 \cdot 10^1 \equiv 1 \cdot 10^0 + 1 \cdot 10^1 (\text{mod } 9)$. By the definition of congruence, $9|(8 - 1) \cdot 10^0 + (3 - 1) \cdot 10^1$, and we know this is true since $10^n \equiv 1 (\text{mod } 9)$. So, an arbitrary integer $n$ can be represented by 
\begin{align*}
n & = a_0 \cdot 10^0 + a_1 \cdot 10^1 + \cdot \cdot \cdot + a_{n-1} \cdot 10^{n-1} + a_n \cdot 10^n \\
& = \sum_{i=1}^{n} a_i \cdot 10^i \\
& \equiv \text{mod } 9.
\end{align*}
\noindent
Now, each term is represented by $10^i$ for some integer $i$, so by part $a$, any positive integer is congruent to the sum of its digits mod $9$. \qedsymbol
\end{enumerate}
\end{problem}

\begin{problem}{7} Section $2.2$ $\#3$ \\

\noindent
Solve the equation, \\
\begin{center}
$x^2 = [1]$ in $\Z_8$. \\
\end{center}
\noindent
The possible equivalence classes of $\Z_8$ are $[0],[1],[2],[3],[4],[5],[6],[7]$. So, we test each equivalence class to see if if satisfies the equation
\begin{center}
$x^2 = [1]$.
\end{center}
\noindent
We will just use an exhaustive method and try all $8$ possibilities,
\renewcommand\labelitemi{\tiny$\bullet$}
\begin{itemize}
\item $[0]$: $[0^2]=[0] \neq 1$ \\
\item $[1]$: $[1^2]=[1] = 1$ \\
\item $[2]$: $[2^2]=[3] \neq 1$ \\
\item $[3]$: $[3^2]=[9]=[1]=1$ \\
\item $[4]$: $[4^2]=[16]=[0]\neq 1$ \\
\item $[5]$: $[5^2]=[25]=[1]=1$ \\
\item $[6]$: $[6^2]=[36]=[4] \neq 1$ \\
\item $[7]$: $[7^2]=[49]=[1] = 1$
\end{itemize}
\noindent
We see that $[1],[3],[5],[7]$ all satisfy the equation $x^2 = [1]$ in $\Z_8$. \qedsymbol
\end{problem}

\newpage

\begin{problem}{8} Section $2.2$ $\#5$ \\

\noindent
Solve the equation, \\
\begin{center}
$x^2 \oplus [3] \varodot x \oplus [2] =[0]$ in $\Z_6$. \\
\end{center}
\noindent
The possible equivalence classes of $\Z_6$ are $[0],[1],[2],[3],[4],[5]$. So, we test each equivalence class to see if if satisfies the equation
\begin{center}
$x^2 \oplus [3] \varodot x \oplus [2] =[0]$.
\end{center}
\noindent
We will just use an exhaustive method and try all $6$ possibilities,
\renewcommand\labelitemi{\tiny$\bullet$}
\begin{itemize}
\item $[0]$: $[0]^2 \oplus [3] \varodot [0] \oplus [2]=[2]\neq [0]$ \\
\item $[1]$: $[1]^2 \oplus [3] \varodot [1] \oplus [2]=[6]= [0]$ \\
\item $[2]$: $[2]^2 \oplus [3] \varodot [2] \oplus [2]=[12]= [0]$ \\
\item $[3]$: $[3]^2 \oplus [3] \varodot [3] \oplus [2]=[20]\neq [0]$ \\
\item $[4]$: $[4]^2 \oplus [3] \varodot [4] \oplus [2]=[30]= [0]$ \\
\item $[5]$: $[5]^2 \oplus [3] \varodot [5] \oplus [2]=[42]= [0]$
\end{itemize}
\noindent
We see that $[1],[2],[4],[5]$ all satisfy the equation $x^2 \oplus [3] \varodot x \oplus [2] =[0]$ in $\Z_6$. \qedsymbol
\end{problem}

\end{document} 