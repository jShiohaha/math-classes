\documentclass{article}

%----------------------------------------------------------------------------------------

\usepackage{listings} % Required for inserting code snippets
\usepackage{geometry}
\geometry{margin=0.7in}
\usepackage[usenames,dvipsnames]{color} % Required for specifying custom colors and referring to colors by name
\usepackage{amssymb}
\usepackage{amsmath}
\usepackage{mathtools}
\usepackage{tikz}
\usepackage{enumerate}

\delimitershortfall-1sp
\newcommand\abs[1]{\left|#1\right|}

\definecolor{DarkGreen}{rgb}{0.0,0.4,0.0} % Comment color
\definecolor{highlight}{RGB}{255,251,204} % Code highlight color

%----------------------------------------------------------------------------------------

\begin{document}

%----------------------------------------------------------------------------------------

\begin{flushleft}
\begin{center}
$\displaystyle \lim_{x \to 2} (x^2 - x) = 2$
\end{center}

We know that $c=2$, $L=2$, $f(x) = x^2 - x$. \\

\vspace{.4cm}
\begin{center}
$\abs{f(x) - L} = \abs{x^2-x-2} < \epsilon \implies \abs{\abs{x-2} \abs{x+1}} < \epsilon$ \\
\end{center}
\vspace{.4cm}

We cannot define $\delta_\epsilon$ in terms of $x$, but $\abs{x-2} = \frac{\epsilon}{\abs{x+1}}$ leaves $\delta_\epsilon$ in terms of $x$. We need to find a fixed $M$ such that $\frac{\epsilon}{M} < \frac{\epsilon}{\abs{x+1}}$. We only care about $x$ \textit{around} $2$, so we can restrict $x$ to the open interval $(1,3)$. 

\begin{center}
\begin{tabular}{c c}
$1<x<3$ & $1<x<3$ \\
$-1<x-2<1$ & $2<x+1<4$ \\
$\abs{x-2}<1$ & $\frac{1}{2}<\frac{1}{\abs{x+1}}<\frac{1}{4}$ \\
\end{tabular}
\end{center}


\subsection*{Proof}
Let $\epsilon > 0$ be given. Let $\delta_\epsilon=min\{1, \frac{\epsilon}{4}\}$. Then $0<\abs{x-2}<\delta_\epsilon$ implies
\begin{center}
\begin{tabular}{l}
(1) $\abs{x-2} < 1 \implies 2<\abs{x+1}<4$ \\
(2) $\abs{f(x) - L} =\abs{x^2-x-2} = \abs{x-2}\abs{x+1} < \frac{\epsilon}{4}\cdot4 = \epsilon$ \\
\end{tabular}
\end{center}

This completes the proof.
\end{flushleft}

\vspace{.75cm}

\begin{flushleft}
\begin{center}
$\displaystyle \lim_{x \to 0} x^2(sin(x) + cos(x)) = 0$
\end{center}
\vspace{.4cm}
We know that $c=0$, $L=0$, $f(x) = x^2(sin(x) + cos(x))$. \\
\vspace{.4cm}
From the combination of the transcendental functions $\abs{sin(x) + cos(x)}$, we know the maximum value that the functions can reach is $\sqrt{2}$. Therefore, we can modify $f(x)$ to be $\sqrt{2} x^2$ since we know that the $\abs{sin(x) + cos(x)}$ part of the function is bounded above. Thus, we have
\begin{center}
\begin{tabular}{r}
$\abs{\sqrt{2} x^2 - 0} < \epsilon$ \\
$\abs{\sqrt{2} x^2} < \epsilon$ \\
$\sqrt{2} x^2 < \epsilon$ \\
$x < \frac{\sqrt{\epsilon}}{\sqrt[4]{2}}$ \\
\end{tabular}
\end{center}

\subsection*{Proof}

Let $\epsilon > 0$ and $\delta_\epsilon=\frac{\sqrt{\epsilon}}{\sqrt[4]{2}}$. Then, $0 < x < \frac{\sqrt{\epsilon}}{\sqrt[4]{2}}$ implies
\begin{center}
$-\frac{\sqrt{\epsilon}}{\sqrt[4]{2}} < x < \frac{\sqrt{\epsilon}}{\sqrt[4]{2}}$ \\
$x^2 < \frac{\epsilon}{\sqrt{2}}$ \\
$\sqrt{2}x^2 < \epsilon$ \\
\end{center}
Thus, $\abs{\sqrt{2}x^2} < \epsilon$. This satisfies the proof because we stated that $sin(x) + cos(x)$ is bounded above by $\sqrt{2}$. By definition of an upper bound, there can never exist a value that exceeds $\abs{sin(x) + cos(x)x^2}$. Then, $\abs{sin(x) + cos(x)x^2} < \abs{x^2\sqrt{2}} < \epsilon$.  
\end{flushleft}

%----------------------------------------------------------------------------------------

\end{document}
