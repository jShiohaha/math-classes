\documentclass{article}
    
%----------------------------------------------------------------------------------------

\usepackage{listings} % Required for inserting code snippets
\usepackage{geometry}
\geometry{margin=0.5in}
\usepackage[usenames,dvipsnames]{color} % Required for specifying custom colors and referring to colors by name
\usepackage{amssymb}
\usepackage{amsmath}
\usepackage{mathtools}
\usepackage{tikz}
\usepackage{enumerate}
\usepackage{environ}

\NewEnviron{myequation}{%
    \begin{equation}
    \scalebox{1.5}{$\BODY$}
    \end{equation}
}

\definecolor{DarkGreen}{rgb}{0.0,0.4,0.0} % Comment color
\definecolor{highlight}{RGB}{255,251,204} % Code highlight color

%----------------------------------------------------------------------------------------

\begin{document}

%----------------------------------------------------------------------------------------

\section*{Problem 5}
\begin{flushleft}
Let's consider all the numbers that are the sum of a rational and a rational mulitple of $\sqrt{2}$. In other words, let's define
\begin{center}
\begin{equation*}
\mathbb{Q}[\sqrt{2}] = \{ r + s \sqrt{2} : r, s \in \mathbb{Q} \}
\end{equation*}
\end{center}
\end{flushleft}
\begin{flushleft}
In this quesiton you'll prove that this is a field (using the usual notions of addition and multiplication). You may assume that $\mathbb{R}$ is a field (as will be our standard assumption from now on).
\begin{enumerate}[a)]
\item Check that addition and multiplication of elements of $\mathbb{Q}[\sqrt{2}]$ results in other elements of $\mathbb{Q}[\sqrt{2}]$. This tells us that we can do arithmetic with things of this form and always end up with things of this form. \\
\begin{flushleft}
\qquad Let's assume that there exists some elements $p$ and $q \in \mathbb{Q}[\sqrt{2}]$. If we prove that both $p + q \in \mathbb{Q}[\sqrt{2}]$ and $pq \in \mathbb{Q}[\sqrt{2}]$, \\
\qquad then we know $\mathbb{Q}[\sqrt{2}]$ is closed under both addition and multiplication.
\begin{center}
\begin{tabular}{r c c l}
(1) & $(r_p + s_p \sqrt{2}) + (r_q + s_q \sqrt{2})$ & $(r_p + s_p \sqrt{2})(r_q + s_q \sqrt{2})$ & (2)
\end{tabular}
\end{center}
\qquad Let's first take a look at addition, denoted by equation (1): \\
\begin{center}
$(r_p + s_p \sqrt{2} + r_q + s_q \sqrt{2})$ \\
\vspace{0.25cm}
$(r_p + r_q + s_p \sqrt{2} + s_q \sqrt{2})$ \\
\vspace{0.25cm}
$(r_p + r_q) + (s_p \sqrt{2} + s_q \sqrt{2})$ \\
\vspace{0.25cm}
$(r_p + r_q) + ((s_p + s_q)\sqrt{2})$ \\
\end{center}
\qquad Now let's say there exists another element $t \in \mathbb{Q}[\sqrt{2}]$. We can redefine $(r_p + r_q)$ to equal $r_t$ because according to the \\
\qquad rules of addition in $\mathbb{Q}$, when $p \in \mathbb{Q}$ and $q \in \mathbb{Q}$, then $(p + q) \in \mathbb{Q}$. By the same notion, $(s_p + s_q)$ to equal $t_s$. Therefore, \\
\qquad  $\mathbb{Q}[\sqrt{2}]$ is closed under addition.

\vspace{.5cm}

\qquad Next let's look at multiplication, denoted by equation (2): \\
\begin{center}
$(r_p + s_p \sqrt{2})(r_q + s_q \sqrt{2})$ \\
\vspace{0.25cm}
$(r_p)(r_q) + (r_q)(s_p \sqrt{2}) + (r_p)(s_q \sqrt{2}) + 2(s_q)(s_p)$ \\
\vspace{0.25cm}
$(r_p r_q) + ((r_q s_p) \sqrt{2}) + ((r_p s_q) \sqrt{2}) + (2s_q s_p)$ \\
\vspace{0.25cm}
$(r_p r_q) + (((r_q s_p) + (r_p s_q)) \sqrt{2}) + (2s_q s_p)$ \\
\vspace{0.25cm}
$(r_p r_q) + (((r_q s_p) + (r_p s_q)) \sqrt{2}) \sqrt{2})$ \\
\vspace{0.25cm}
$(r_p r_q + 2s_q s_p) + (((r_q s_p) + (r_p s_q)) \sqrt{2})$ \\
\end{center}
\qquad Just as we saw with addition, we can redefine $(r_p r_q + 2s_q s_p)$ and $((r_q s_p) + (r_p s_q))$ which will both still be in $\mathbb{Q}$. \\
\qquad Therefore, $\mathbb{Q}[\sqrt{2}]$ is closed under multiplication. \\
\end{flushleft}

\item Check that negatives and reciprocals of elements of $\mathbb{Q}[\sqrt{2}]$ are in $\mathbb{Q}[\sqrt{2}]$. \\

\begin{flushleft}
\qquad Let's assume that there exists some element $p \in \mathbb{Q}[\sqrt{2}]$. If we prove that both $-p \in \mathbb{Q}[\sqrt{2}]$ and $p^{-1} \in \mathbb{Q}[\sqrt{2}]$, \\
\qquad then we know $\mathbb{Q}[\sqrt{2}]$ is closed to both the negative and inverse of elements in $\mathbb{Q}[\sqrt{2}]$.
\begin{center}
\begin{tabular}{r c c l}
(1) & $-(r_p + s_p \sqrt{2})$ & $(r_p + s_p \sqrt{2})^{-1}$ & (2)
\end{tabular}
\end{center}
\qquad Let's first take a look at negation, denoted by equation (1): \\
\begin{center}
$(-1)(r_p + s_p \sqrt{2})$ \\
\vspace{0.25cm}
$(-1)(r_p) + (-1)(s_p \sqrt{2})$ \\
\vspace{0.25cm}
$(-r_p) + ((-s_p) \sqrt{2})$\\
\end{center}
\qquad Now let's say there exists another element $q \in \mathbb{Q}[\sqrt{2}]$. We can redefine $(-r_p)$ to equal $r_q$. By the same notion, $(-s_p)$ \\
\qquad can be redefined to $q_s$. $r_q + s_q \sqrt{2} \in \mathbb{Q}[\sqrt{2}]$.

\vspace{.5cm}

\qquad Next let's look at reciprocal, denoted by equation (2). Note, $p^{-1} = 1/p$. Additionally, we can multiply by the \\
\qquad opposite reciprocal in order to split the numerator with a common denominator.\\

\begin{center}
$(r_p + s_p \sqrt{2})^{-1}$ \\
\vspace{0.25cm}
$\frac{1}{(r_p + s_p \sqrt{2})}$ \\
\vspace{0.25cm}
$\frac{1}{(r_p + s_p \sqrt{2})} \cdot \frac{r_p-s_p\sqrt{2}}{r_p-s_p\sqrt{2}}$ \\
\vspace{0.25cm}
$\frac{r_p-s_p\sqrt{2}}{r_p^2 - 2s_p^2}$ \\
\vspace{0.25cm}
$\frac{r_p}{r_p^2 - 2s_p^2} - \frac{s_p\sqrt{2}}{r_p^2 - 2s_p^2}$ \\
\vspace{0.25cm}
$\frac{r_p}{r_p^2 - 2s_p^2} + \frac{s_p}{2s_p^2 - r_p^2}\sqrt{2}$ \\
\end{center}
\qquad Like with each of the situations before, we can redefine $\frac{r_p}{r_p^2 - 2s_p^2}$ and $\frac{s_p}{2s_p^2 - r_p^2}$ to be $r_q$ and $s_q$ respectively. Both will \\ 
\qquad still be in $\mathbb{Q}[\sqrt{2}]$. Therefore, the reciprocal is closed in $\mathbb{Q}[\sqrt{2}]$. \\
\end{flushleft}

\item Explain briefly why axioms, such as A1, whose quantifiers are purely "$\forall$" are automaticlly true for  $\mathbb{Q}[\sqrt{2}]$ since they are true for $\mathbb{R}$. Pick out which axioms this argument works for.

\vspace{.5cm}

\qquad For $\mathbb{Q}[\sqrt{2}]$ to be a field, it must be a set together with two binary operations called \emph{addition} and \emph{multiplication} that \\
\qquad satisfies the 11 axioms as previously defined. The axioms containing "$\forall x$"as defined in $\mathbb{R}$ automatically hold "$\forall$" in \\ 
\qquad $\mathbb{Q}[\sqrt{2}]$ because any subset of $\mathbb{Q}$ is also a subset of $\mathbb{R}$ since $\mathbb{Q} \subset \mathbb{R}$. However, some of the axioms contain "$\exists x$" in \\
\qquad them, which do not automatically extend to $\mathbb{Q}[\sqrt{2}]$ because although $\mathbb{Q} \subset \mathbb{R}$, $\mathbb{R} \not \subset \mathbb{Q}$. Such axioms are A2, A3, A7, \\
\qquad A8, A11. Therefore, there will be some elements of $\mathbb{R}$ that do not exist in $\mathbb{Q}$. Those axioms must therefore be \\ \qquad proved. Such axioms are A1, A4, A5, A6, A9, and A10. 

\item Prove that $\mathbb{Q}[\sqrt{2}]$ is a field, using your results from earlier parts.

\vspace{.5cm}

\qquad As stated in part (c), the axioms that do not automatically hold for $\mathbb{Q}[\sqrt{2}]$ are A1, A4, A5, A6, A9, and A10. In \\
\qquad  part (a), we proved that there was closure under addition and multiplication, satisfying axioms A1 and A6. In part \\
\qquad (b), we proved the existence of both additive and multiplicative inverses, satisfying axioms A5 and A10. The only \\
\qquad ones left to prove are A4 and A9. 

\vspace{.5cm}

\qquad To prove A4, we need to show there exists an additive identity such that $\forall a \in \mathbb{F}$, $a+0=a$ In this case, we can \\
\qquad redefine $r + s\sqrt{2}$ as $a$.

\begin{center}
$(r + s \sqrt{2}) + 0 = (r + s \sqrt{2})$\\
$((r + s \sqrt{2}) + (-(r + s \sqrt{2}))) + 0 = (r + s \sqrt{2}) + (-(r + s \sqrt{2})))$
\end{center}

\qquad From here, we can use the A5 (since we proved it in part (b)), which states the existance of an additive inverse such \\
\qquad that $a + (-a)=0$. We redefined $(r + s \sqrt{2})$ as $a$, so we know the above equation simplifies to $0=0$. Thus, Axiom A4\\
\qquad holds. 

\vspace{.5cm}

\qquad To prove A9, we need to show there exists an multiplicative identity such that $\forall a \in \mathbb{F}$, with $a \neq 0$, $a\cdot 1=a$. Again, \\
\qquad we can redefine $r + s\sqrt{2}$ as $a$.

\begin{center}
$1 \cdot (r + s \sqrt{2}) = (r + s \sqrt{2})$\\
\end{center}

\qquad From here, we can use the A10 (since we proved it in part (b)), which states the existance of an multiplicative \\
\qquad inverse. We redefined $(r + s \sqrt{2})$ as $a$, so we know the above equation implifies to $1=1$. Thus, Axiom A9 holds.

\vspace{.5cm}

\qquad We have now proved that all 11 axioms hold for $\mathbb{Q}[\sqrt{2}]$ under the binary operations of \emph{addition} and \emph{multiplication}. \\
\qquad Thus, $\mathbb{Q}[\sqrt{2}]$ is a field.
\end{enumerate}
\end{flushleft}



%----------------------------------------------------------------------------------------

\end{document}
