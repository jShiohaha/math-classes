\documentclass{article}

%----------------------------------------------------------------------------------------

\usepackage{listings} % Required for inserting code snippets
\usepackage{geometry}
\geometry{margin=0.7in}
\usepackage[usenames,dvipsnames]{color} % Required for specifying custom colors and referring to colors by name
\usepackage{amssymb}
\usepackage{amsmath}
\usepackage{mathtools}
\usepackage{tikz}
\usepackage{enumerate}
\usepackage[T1]{fontenc}

\delimitershortfall-1sp
\newcommand\abs[1]{\left|#1\right|}

\definecolor{DarkGreen}{rgb}{0.0,0.4,0.0} % Comment color
\definecolor{highlight}{RGB}{255,251,204} % Code highlight color

%----------------------------------------------------------------------------------------

\begin{document}

%----------------------------------------------------------------------------------------

\subsubsection*{Group Problem 5.5}
\begin{flushleft}
Recall that a function $f \in \mathbb{R}^2$ is called a polynomial if there is a non-negative integer $n$ (called "the degree" of $f$) and real numbers $c0, c1, ..., cn$, such that for every $x \in \mathbb{R}$.
\begin{center}
$f(x) = \sum_{i=0}^{n} c_ix^i$
\end{center}
Use induction and the theorem about combining the limits to prove that every polynomial is continuous on $\mathbb{R}$. (Note: the results about combining limits apply to at most two functions at a time. Also don't forget that you must check the case when $n = 0$, specifically, when $f$ is a constant function).
\end{flushleft}

\subsubsection*{Proof}
\begin{flushleft}
First, we'll prove the base case of $n=1$. When $n=1$,
\begin{center}
$\sum_{i=0}^{1} c_ix^i = c_0 + c_1x$.
\end{center}
Since the limit of the resulting function is constructed based on the two limit definitions of
\begin{center}
\begin{tabular}{c c}
$\lim_{x \to a} c = c$ & $\lim_{x \to a} x = a$,
\end{tabular}
\end{center}
we can combine them to show that $c_0 + c_1x$ is continuous if the limit exists for $\lim_{x \to a} c_0 + c_1x$. Thus, we know the polynomial of $n=1$ is also continuous.

\vspace{1.5cm}

Next, for the induction step, we assume a polynomial $P_n$ is continuous as defined $f(x) = c_0 + c_1x + ... + c_nx^n$. Let $g(x)=c_{n+1}x^{n+1}$. We know $c_1x$ and $c_nx^n$ are continuous from our initial assumption about $P_n$, so we can see that $c_{n+1}(x^n \cdot x)$ is also continuous for any $c \in \mathbb{R}$ as a result of 

\begin{center}
\begin{tabular}{l}
$\lim_{x \to a} (f \cdot g)(x)$ \\
$\lim_{x \to a} (cf)(x)$, $\forall c \in \mathbb{R}$. \\
\end{tabular}
\end{center}

Since $g(x)$ is also continuous, we know by $\lim_{x \to a} (f + g)(x)$ that $P_{n+1}$ is also continuous. This completes the proof.
\end{flushleft}

%----------------------------------------------------------------------------------------

\end{document}
